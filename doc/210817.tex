% beamer 16:9
\documentclass[aspectratio=169, 9pt, xcolor=table]{beamer}
% suppress the navigation bar
\beamertemplatenavigationsymbolsempty
% add frame number to footline
\setbeamertemplate{footline}[frame number]
% set frame color
% \usecolortheme[named=cyan]{structure}
% set font serif only math
% \usefonttheme[onlymath]{serif}
% set font serif all
\usefonttheme{serif}
\usepackage{subcaption}
% subfile - making directories for slides
\usepackage{subfiles}
% wrapping figures with text
\usepackage{wrapfig}
% table 과 figure 에 번호 붙이기
\setbeamertemplate{caption}[numbered]
% math package
\usepackage{amsmath, amsthm, amssymb}
% code 넣을 때 frame 으로 사용
\usepackage{listings}
% code 넣기
% \usepackage{minted}
% table 에 rule 넣기
\usepackage{booktabs}
% talbe 에 multirow 적용
\usepackage{multirow}
\usepackage{animate}
\usepackage{forarray}
% remove font size error
\usepackage{lmodern}
% hyperlinks
\usepackage{hyperref}
% foreach and other making tools with if and for loop
\usepackage{pgffor}
\usepackage{calc}
\usepackage{ifthen}
% section 만들기
\usepackage{xparse}% http://ctan.org/pkg/xparse
\usepackage{etoolbox}% http://ctan.org/pkg/etoolbox
\usepackage[utf8]{inputenc}
\bibliographystyle{apalike}

\setbeamertemplate{bibliography entry title}{}
\setbeamertemplate{bibliography entry location}{}
\setbeamertemplate{bibliography entry note}{}
\usepackage{kotex}
\usepackage{seqsplit} % for automatic line breaks

% \usepackage[
% backend=biber,
% style=numeric,
% sorting=ynt
% ]{biblatex}
\newcommand{\tred}[1]{\textcolor{red}{#1}}

\DeclareMathOperator*{\argmin}{argmin}
\DeclareMathOperator{\Tr}{Tr}

\AtBeginSection[]
{
\begin{frame}
\vfill
\centering
\usebeamerfont{section title}\insertsection
\vfill
\end{frame}
}

% set theme color
\definecolor{smurf}{rgb}{0.14, 0.39, 0.78}  %%% <- Change it into the color you want
\usecolortheme[named=smurf]{structure}      %%% <- Change it into the color you want

% set title
\title{CPE with Agent Based Modeling}
\author{Nathan Cho}
\institute{Computational Science and Engineering}
\date{\today}
%%%%%%%%%%%%%%%%%%%%%%%%%%%%%%%%%%%%%%%%%%%%%%%%%%%%%%%%%%%%%%%%%%%%%%%%%%%%
\usepackage[style=verbose,backend=biber]{biblatex}
\addbibresource{abmBib.bib}
\begin{document}

\begin{frame}
    \titlepage
\end{frame}
%%%%%%%%%%%%%%%%%%%%%%%%%%START%%%%%%%%%%%%%%%%%%%%%%%%%%%%%%%%%%%%%%%%%%%%%%


\begin{frame}
    \frametitle{Setting}
    \begin{figure}[H]
        \centering
        \includegraphics[width=.5\textwidth]{../../../ref/figures/blueprint.png} %must add a space here
        \caption{Blueprint}
    \end{figure}
    2 HCWs roam ICU-A and ICU-B. (Randomly divide the beds)
\end{frame}

\begin{frame}
    \frametitle{Calibration of transmission probability and isolation factor}
    
    Recall:
    
    $\textit{prob transmission}$ is the probability that the HCW infects the patient or environment (or the patient contaminates the HCW)
    
    $\textit{isolation factor}$ is the number multiplied to the prob transmission for patients in isolated beds.


    \begin{figure}[H]
        \centering
        \includegraphics[width=.3\textwidth]{../../../result/batchrun/trans_isol.png} %must add a space here
        \caption{Calibration}
    \end{figure}
    Note that the isolation factor seems to have a greater effect.

    We choose $\textit{prob transmission} = .075$ and $\textit{isolation factor} = .15$

    
    
    %Example of actual response:
    %$\{A2,A6,A9,A13,B8,B12,B13,A15\}$
    %$\{A1,A2,A4,A8,A9,A11,A14,A15,B3,B6,B8,B9,B10,B11,B12,B14\}$
    %$\{A1,A4,A12,A11,B9,B12,B13,B2,B1\}$

\end{frame}



\begin{frame}
    \frametitle{Controls}
    
    \begin{itemize}
        \item Environment cleansing
        \item HCW route (eg. visit isolated beds only? or both?)
        \item Handwashing probability (from literature, fixed at .9)
        \item Isolation of infected patients
    \end{itemize}
\end{frame}

\begin{frame}
    \frametitle{Questions}
    
    \begin{itemize}
        \item Is there a better way to model isolated beds?
        \item HCWs routes should also depend on where the patients are. Is this observable?
        
    \end{itemize}
\end{frame}
% \begin{frame}
%     \frametitle{Attempt to view the long run equilibrium}
%     \begin{figure}[H]
%         \centering
%         \includegraphics[width=.5\textwidth]{../../../ref/figures/abm_longrun.png} %must add a space here
%         \caption{Sample run}
%     \end{figure}
%     \begin{itemize}
%         \item Notice that in the long run, almost every bed that the HCWs visit ends up being sick
        
%     \end{itemize}
% \end{frame}

% \begin{frame}
%     \frametitle{Some ideas for control}
%     \begin{itemize}
%         \item Set routes and find the optimal partition
%         \item Make best use of isolated beds
%         \item Environment cleansing
%         \item Max handwashing; but this would imply, in theory, no infection within the ICU
%         \item \textit{Need a baseline to compare the strategies}
        
%     \end{itemize}
% \end{frame}


\end{document}



